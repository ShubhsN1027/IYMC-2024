\documentclass[12pt,a4paper]{article}

\usepackage{amsmath, amssymb, amsthm, amsfonts}
\usepackage{fullpage}
\usepackage{fancyhdr}
\usepackage{graphicx}

% Set the header style
\pagestyle{fancy}
\fancyhf{}
\fancyhead[L]{IYMC 2024 Qualification Round}
\fancyhead[R]{Solution C}
\fancyfoot{}
\renewcommand{\headrulewidth}{0.4pt}
% The above sets up the page header and footer.

% Increase the vertical space between header and text
\setlength{\headsep}{1cm}

\begin{document}

\title{Solution: Problem C}
\date{}
\maketitle
\thispagestyle{fancy} % Apply the fancy page style to the title page as well.



\noindent\textbf{Given Problem:} Determine the numerical value of the following expression without the use of a calculator:

\[
\left( \sqrt{2} + (3^2)^{\frac{1}{4}} + \sum_{m=1}^{3} \left(\frac{1}{m!} - \sqrt{m}\right) \right) \cdot \left( 2^{\log_2(8)} + \frac{1}{2^3} - \prod_{k=1}^{8}\left(1 + \frac{1}{k}\right) \right).
\]

We are informed the final answer is \(-\frac{7}{12}\). We will verify this step-by-step, justifying each simplification thoroughly.

% Begin the solution
\section*{Detailed solution for Problem C...}

1. \textbf{We first rewrite the main expression:}
\bigskip

   Let
   \[
   A = \left(\sqrt{2} + (3^2)^{1/4} + \sum_{m=1}^{3} \left(\frac{1}{m!} - \sqrt{m}\right)\right)
   \]
   and
   \[
   B = \left(2^{\log_2(8)} + \frac{1}{2^3} - \prod_{k=1}^{8} \left(1 + \frac{1}{k}\right)\right).
   \]

   Our target is to compute:
   \[
   A \times B.
   \]

2. \textbf{We then simplify \((3^2)^{1/4}\):}

   We have:
   \[
   (3^2)^{1/4} = 9^{1/4}.
   \]

   Since \(9 = 3^2\),
   \[
   9^{1/4} = (3^2)^{1/4} = 3^{2/4} = 3^{1/2} = \sqrt{3}.
   \]

   Therefore:
   \[
   (3^2)^{1/4} = \sqrt{3}.
   \]

3. \textbf{Now, we substitute this into \(A\):}

   Now:
   \[
   A = \left(\sqrt{2} + \sqrt{3}\right) + \sum_{m=1}^{3}\left(\frac{1}{m!} - \sqrt{m}\right).
   \]




   4. \textbf{We evaluate the summation \(\sum_{m=1}^{3}\frac{1}{m!}\):}

   Computing each factorial:
   \[
   1! = 1, \quad 2! = 2, \quad 3! = 6.
   \]
   Thus:
   \[
   \frac{1}{1!} = 1,\quad \frac{1}{2!} = \frac{1}{2},\quad \frac{1}{3!} = \frac{1}{6}.
   \]

   Summing these:
   \[
   1 + \frac{1}{2} + \frac{1}{6}.
   \]

   Finding a common denominator (6):
   \[
   1 = \frac{6}{6}, \quad \frac{1}{2} = \frac{3}{6}, \quad \frac{1}{6}=\frac{1}{6}.
   \]

   Adding them:
   \[
   \frac{6}{6} + \frac{3}{6} + \frac{1}{6} = \frac{6+3+1}{6} = \frac{10}{6} = \frac{5}{3}.
   \]

   Therefore:
   \[
   \sum_{m=1}^{3}\frac{1}{m!} = \frac{5}{3}.
   \]

5. \textbf{We now evaluate \(\sum_{m=1}^{3}\sqrt{m}\):}

   \[
   \sqrt{1} = 1, \quad \sqrt{2}=\sqrt{2},\quad \sqrt{3}=\sqrt{3}.
   \]

   Thus:
   \[
   \sum_{m=1}^{3}\sqrt{m} = 1 + \sqrt{2} + \sqrt{3}.
   \]

6. \textbf{We combine the terms in the summation:}

   We have:
   \[
   \sum_{m=1}^{3}\left(\frac{1}{m!} - \sqrt{m}\right) = \sum_{m=1}^{3}\frac{1}{m!} - \sum_{m=1}^{3}\sqrt{m}.
   \]

   Substitute the values:
   \[
   = \frac{5}{3} - (1 + \sqrt{2} + \sqrt{3}).
   \]

7. \textbf{The substitute back into \(A\):}

   Recall:
   \[
   A = (\sqrt{2} + \sqrt{3}) + \left(\frac{5}{3} - (1 + \sqrt{2} + \sqrt{3})\right).
   \]

   Distribute the subtraction:
   \[
   A = (\sqrt{2} + \sqrt{3}) + \frac{5}{3} - 1 - \sqrt{2} - \sqrt{3}.
   \]

   Combine like terms:
   \[
   \sqrt{2} - \sqrt{2} = 0, \quad \sqrt{3} - \sqrt{3} = 0.
   \]

   All the radicals cancel out. Hence, we are left with:
   \[
   A = \frac{5}{3} - 1.
   \]

   Convert 1 to \(\frac{3}{3}\):
   \[
   A = \frac{5}{3} - \frac{3}{3} = \frac{2}{3}.
   \]

   Hence:
   \[
   A = \frac{2}{3}.
   \]

8. \textbf{We now evaluate the second bracket \(B\):}

   Recall:
   \[
   B = 2^{\log_2(8)} + \frac{1}{2^3} - \prod_{k=1}^{8}\left(1 + \frac{1}{k}\right).
   \]

   First, simplify \(2^{\log_2(8)}\):
   Since \(8=2^3\), \(\log_2(8)=3\). Thus:
   \[
   2^{\log_2(8)} = 2^3 = 8.
   \]

   Next, \(\frac{1}{2^3} = \frac{1}{8}\).

   \bigskip

   Now, we consider the product:
   \[
   \prod_{k=1}^{8}\left(1+\frac{1}{k}\right) = \prod_{k=1}^{8}\frac{k+1}{k}.
   \]

   Write out a few terms:
   \[
   = \frac{2}{1}\times\frac{3}{2}\times\frac{4}{3}\times\frac{5}{4}\times\frac{6}{5}\times\frac{7}{6}\times\frac{8}{7}\times\frac{9}{8}.
   \]

   Most terms cancel out in a telescoping manner:
   \[
   = \frac{9}{1} = 9.
   \]

   Therefore:
   \[
   B = 8 + \frac{1}{8} - 9.
   \]

   Combine \(8-9=-1\):
   \[
   B = -1 + \frac{1}{8}.
   \]

   Convert \(-1 = -\frac{8}{8}\):
   \[
   B = -\frac{8}{8} + \frac{1}{8} = -\frac{7}{8}.
   \]

   Thus:
   \[
   B = -\frac{7}{8}.
   \]

   \bigskip

9. \textbf{We now multiply \(A\) and \(B\):}

   We had found:
   \[
   A = \frac{2}{3}, \quad B = -\frac{7}{8}.
   \]

   Multiplying:
   \[
   A \times B = \frac{2}{3}\times\left(-\frac{7}{8}\right) = \frac{2\times(-7)}{3\times8} = \frac{-14}{24}.
   \]

   We then simplify by dividing numerator and denominator by 2:
   \[
   = \frac{-7}{12}.
   \]

   Therefore:
   \[
   \boxed{-\frac{7}{12}}.
   \]


   \bigskip

   \subsection*{Mathematical concepts involved in Problem C...}
   \begin{itemize}
      \item \textbf{Manipulation of fractional expressions and simplification of factorial-based fractions.}
      \item \textbf{Use of logarithm properties, such as \(2^{\log_2(8)} = 8\).}
      \item \textbf{Telescoping products to simplify complex products into simpler fractions.}
      \item \textbf{Careful handling of sums and products involving factorials and radicals.}
   \end{itemize}
   
   \bigskip
   \noindent\hrulefill
   \begin{center}
   \textbf{************* END OF SOLUTION: C *************}
   \end{center}
   \hrulefill
   
   \end{document}
