\documentclass[12pt,a4paper]{article}

\usepackage{amsmath, amssymb, amsfonts}
\usepackage{fullpage}
\usepackage{fancyhdr}
\usepackage{graphicx}
\usepackage{array}

% Set the header
\pagestyle{fancy}
\fancyhf{}
\fancyhead[L]{IYMC 2024 Qualification Round}
\fancyhead[R]{Solution A}
\fancyfoot{}
\renewcommand{\headrulewidth}{0.4pt}
\setlength{\headsep}{1cm}

\begin{document}

\title{Solution: Problem A}
\date{}
\maketitle
\thispagestyle{fancy}



\noindent\textbf{The Problem Statement:} \\
We are given two sequences of numbers:

\[
\begin{array}{c|cccccc}
n & 1 & 2 & 3 & 4 & 5 & 6 \\ \hline
a_n & 2 & 5 & 10 & 17 & 26 & 37 \\
b_n & 1 & 2 & 8 & 48 & 384 & 3840
\end{array}
\]

The problem states to analyze each sequence, continue the pattern, and find a closed-form (an explicit formula) for the \( n \)-th term of each sequence, \( a_n \) and \( b_n \).

\section*{Detailed solution for Problem A...}

\subsection*{Part 1: Deriving the Formula for $a_n$}

1. \textbf{We inspect the given terms for $a_n$:} \\
The sequence is:
\[
a_1=2,\quad a_2=5,\quad a_3=10,\quad a_4=17,\quad a_5=26,\quad a_6=37.
\]
We have six terms. To find a pattern, a common method and my first instinct was to look at the differences between consecutive terms.
\bigskip

2. \textbf{First Differences \(\Delta a_n\):} \\
We compute \(\Delta a_n = a_{n+1}-a_n\) for each pair of consecutive terms:
\[
a_2 - a_1 = 5 - 2 = 3,\quad a_3 - a_2 = 10 - 5 = 5,\quad a_4 - a_3 = 17 - 10 = 7,\quad a_5 - a_4 = 26 - 17 = 9,\quad a_6 - a_5 = 37 - 26 = 11.
\]

So the first differences are:
\[
\Delta a_n: 3, 5, 7, 9, 11, \dots
\]

We observe that these differences form an arithmetic progression with a common difference of 2.
\bigskip

3. \textbf{Second Differences \(\Delta^2 a_n\):} \\
Now, the second differences are checked which are the differences of the first differences:
\[
5-3=2,\quad 7-5=2,\quad 9-7=2,\quad 11-9=2.
\]

All second differences are constant and equal to 2. A sequence with a constant second difference indicates that \( a_n \) can be expressed as a quadratic polynomial in \( n \).

4. \textbf{We will assume a Quadratic Form:} \\
Since we have constant second differences, we assume:
\[
a_n = A n^2 + B n + C,
\]
where \( A, B, C \) are constants we need to determine.
\bigskip

5. \textbf{Plugging in the known terms:} \\
Using the first three terms of the sequence (\( a_1, a_2, a_3 \)) to form three equations in terms of \( A, B, C \).

\bigskip
- For \( n=1 \), \( a_1=2 \):
\[
A(1)^2+B(1)+C = A + B + C = 2.
\]
So:
\[
A + B + C = 2. \quad (1)
\]

- For \( n=2 \), \( a_2=5 \):
\[
A(2^2)+B(2)+C = 4A + 2B + C = 5.
\]
So:
\[
4A + 2B + C = 5. \quad (2)
\]

- For \( n=3 \), \( a_3=10 \):
\[
A(3^2)+B(3)+C = 9A + 3B + C = 10.
\]
So:
\[
9A + 3B + C = 10. \quad (3)
\]

Now we have three linear equations:
\[
(1)\; A+B+C=2
\]
\[
(2)\; 4A+2B+C=5
\]
\[
(3)\; 9A+3B+C=10
\]

6. \textbf{We now solve the system for \( A, B, C \):} \\
To solve, we eliminate variables step-by-step:

\bigskip

\textbf{Step A: Subtract Equation (1) from Equation (2):} \\
\((2)-(1)\):
\[
(4A+2B+C) - (A+B+C) = 5 - 2.
\]
On the left:
- The \(C\) cancels.
- \(4A - A = 3A\).
- \(2B - B = B\).
On the right: \(5-2=3\).

Thus:
\[
3A + B = 3. \quad (4)
\]

\textbf{Step B: Subtract Equation (1) from Equation (3):} \\
\((3)-(1)\):
\[
(9A+3B+C) - (A+B+C) = 10 - 2.
\]
On the left:
- \(C\) cancels.
- \(9A - A = 8A\).
- \(3B - B = 2B\).
On the right: \(10-2=8\).

We have:
\[
8A + 2B = 8. \quad (5)
\]

\textbf{Step C: Simplify Equation (5):} \\
Divide (5) by 2:
\[
4A + B =4. \quad (6)
\]

Now compare (4) and (6):
From (4): \(3A+B=3\).
From (6): \(4A+B=4\).
\bigskip

\textbf{Step D: Subtract (4) from (6):} \\
\[
(4A+B) - (3A+B)=4 - 3.
\]
Left side:
- \(B\) cancels.
- \(4A-3A= A\).
Right side:
\[
4-3=1.
\]

So:
\[
A=1.
\]

With \( A=1 \), substitute into (4) \(3A+B=3\):
\[
3(1)+B=3 \implies 3+B=3 \implies B=0.
\]

Now \( A=1,B=0 \). From (1): \(A+B+C=2\):
\[
1+0+C=2 \implies C=1.
\]

Thus:
\[
A=1,\; B=0,\; C=1.
\]

7. \textbf{Final Quadratic Formula for \( a_n \):} \\
\[
a_n = n^2 + 1.
\]

\[
\boxed{a_n = n^2 + 1.}
\]

\subsection*{Part 2: Deriving the Formula for $b_n$}

1. \textbf{We first inspecting the given terms for $b_n$:} \\
The sequence is:
\[
b_1=1,\quad b_2=2,\quad b_3=8,\quad b_4=48,\quad b_5=384,\quad b_6=3840.
\]

The terms grow very fast which is key hint. We will hendforth look at the ratio of consecutive terms.

\bigskip

2. \textbf{Check Ratios \(\frac{b_{n+1}}{b_n}\):} \\
Compute:
\[
\frac{b_2}{b_1}=\frac{2}{1}=2,\quad \frac{b_3}{b_2}=\frac{8}{2}=4,\quad \frac{b_4}{b_3}= \frac{48}{8}=6,\quad \frac{b_5}{b_4}=\frac{384}{48}=8,\quad \frac{b_6}{b_5}=\frac{3840}{384}=10.
\]

The ratios are:
\[
2,4,6,8,10,\dots
\]

These ratios form an arithmetic sequence starting at 2 and increasing by 2 each time. The \(k\)-th ratio is \(2k\).

\bigskip

3. \textbf{Constructing $b_n$ using ratios:} \\
Start with \( b_1=1 \):

- To get \( b_2 \): multiply \( b_1 \) by 2:
\[
b_2 = 1\times 2.
\]

- To get \( b_3 \): multiply \( b_2 \) by 4:
\[
b_3 = (1\times 2)\times 4 = 1\times 2\times 4.
\]

- To get \( b_4 \): multiply \( b_3 \) by 6:
\[
b_4 = (1\times 2\times 4)\times 6 = 1\times 2\times 4\times 6.
\]

- To get \( b_5 \): multiply \( b_4 \) by 8:
\[
b_5 = (1\times 2\times 4\times 6)\times 8 = 1\times2\times4\times6\times8.
\]

\bigskip

4. \textbf{Visualizing the Pattern ("b Triangle"):} \\
\bigskip
We arrange factors in a triangular form:

\[
b_1: \; 1
\]

\[
b_2: \; 1 \;\times 2
\]

\[
b_3: \; 1 \;\times 2 \;\times 4
\]

\[
b_4: \; 1 \;\times 2 \;\times 4 \;\times 6
\]

\[
b_5: \; 1 \;\times 2 \;\times 4 \;\times 6 \;\times 8
\]

Each $b_n$ (for $n>1$) is formed by multiplying together even numbers starting from 2.

\bigskip

5. \textbf{Expressing $b_n$ in General Form:} \\
For $n>1$:
\[
b_n = 1\times 2\times 4\times 6 \times \cdots \times [2(n-1)].
\]

There are $(n-1)$ even factors. Write them as $2k$:
\[
b_n = \prod_{k=1}^{n-1}(2k).
\]

Factor out 2 from each term:
\[
b_n = 2^{n-1}(1\cdot 2 \cdot 3 \cdots (n-1)) = 2^{n-1}(n-1)!.
\]

Check for a few terms matches perfectly.

\[
\boxed{b_n = 2^{n-1}(n-1)!}.
\]

\subsection*{Mathematical concepts involved in Problem A...}

\begin{itemize}
\item \textbf{Arithmetic Progressions (AP):} Used to identify patterns in the first differences of $a_n$ and the sequence of ratios for $b_n$.
\item \textbf{Constant Second Differences \& Quadratic Sequences:} For $a_n$, the constant second difference confirmed it can be represented by a quadratic polynomial.
\item \textbf{Solving Linear Equations:} Found $A,B,C$ for $a_n$ by forming and solving a system of linear equations.
\item \textbf{Factorials:} To recognize the product pattern for $b_n$ led to a factorial expression $(n-1)!$ and a power of 2.
\item \textbf{Decomposition into Basic Factors:} For $b_n$, factoring out 2 from every even factor gave a neat closed form.
\end{itemize}

\subsection*{Final Answers}

\[
\boxed{a_n = n^2 + 1,\quad b_n = 2^{n-1}(n-1)!.}
\]

\bigskip
\noindent\hrulefill
\begin{center}
\textbf{************* END OF SOLUTION: A *************}
\end{center}
\hrulefill

\end{document}
