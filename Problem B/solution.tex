\documentclass[12pt,a4paper]{article}

\usepackage{amsmath, amssymb, amsthm, amsfonts}
\usepackage{fullpage}
\usepackage{fancyhdr}
\usepackage{graphicx}

% Set the header style
\pagestyle{fancy}
\fancyhf{}
\fancyhead[L]{IYMC 2024 Qualification Round}
\fancyhead[R]{Solution B}
\fancyfoot{}
\renewcommand{\headrulewidth}{0.4pt}

% Increase vertical space between header and text if needed
\setlength{\headsep}{1cm}

\begin{document}

\title{Solution: Problem B}
\date{}
\maketitle
\thispagestyle{fancy}


\noindent\textbf{Given Problem:} Find all $x \in \mathbb{R}$ that solve the equation:
\[
x^4 + x^2 - x - 1 = 1 - x - x^2 - x^4.
\]


\subsection*{Deatiled solution for Problem B...}

1. \textbf{We start with the given equation:}
\[
x^4 + x^2 - x - 1 = 1 - x - x^2 - x^4.
\]

2. \textbf{Bringing all the terms to one side:}  
Subtract $(1 - x - x^2 - x^4)$ from both sides:
\[
x^4 + x^2 - x - 1 - (1 - x - x^2 - x^4) = 0.
\]

Distributing the minus sign:
\[
x^4 + x^2 - x - 1 - 1 + x + x^2 + x^4 = 0.
\]

3. \textbf{We then combine like terms:}
- $x^4$ terms: $x^4 + x^4 = 2x^4$.
- $x^2$ terms: $x^2 + x^2 = 2x^2$.
- $x$ terms: $-x + x = 0$.
- Constants: $-1 -1 = -2$.

So we have:
\[
2x^4 + 2x^2 - 2 = 0.
\]

4. \textbf{Factoring out the common factor:}  
Divide through by 2:
\[
x^4 + x^2 - 1 = 0.
\]

5. \textbf{We then use a substitution to solve the quartic:}  
Let $y = x^2$. Then $y \geq 0$ for real $x$. Substitute:
\[
y^2 + y - 1 = 0.
\]

We have reduced to a quadratic equation in $y$.

\bigskip

6. \textbf{Now, we solve the quadratic equation in $y$:}

\bigskip

Using the quadratic formula for $ay^2+by+c=0$: $y=\frac{-b \pm \sqrt{b^2-4ac}}{2a}$.

\bigskip

Here, $a=1$, $b=1$, $c=-1$. Thus:
\[
y = \frac{-1 \pm \sqrt{1^2 -4(1)(-1)}}{2} = \frac{-1 \pm \sqrt{1+4}}{2} = \frac{-1 \pm \sqrt{5}}{2}.
\]

So:
\[
y_1 = \frac{-1 + \sqrt{5}}{2}, \quad y_2 = \frac{-1 - \sqrt{5}}{2}.
\]

\bigskip

7. \textbf{We now need to check which $y$ is valid:}  
Since $y=x^2 \geq 0$, discard negative solutions.

Evaluate sign:
- $\sqrt{5} \approx 2.236$, $-1+\sqrt{5}>0$, so $y_1 = \frac{-1+\sqrt{5}}{2}>0$.
- $-1-\sqrt{5}<0$, so $y_2$ is negative.

Discard $y_2$. Thus:
\[
y = \frac{-1 + \sqrt{5}}{2}.
\]

\bigskip

8. \textbf{Find $x$:}  
Since $y = x^2$, we have:
\[
x^2 = \frac{-1+\sqrt{5}}{2}.
\]

Take the square root (considering both positive and negative roots):
\[
x = \pm \sqrt{\frac{-1 + \sqrt{5}}{2}}.
\]

\bigskip

9. \textbf{Final Solution:}
\[
\boxed{x = \pm \sqrt{\frac{-1 + \sqrt{5}}{2}}}.
\]

These are all the real solutions to the given equation.

\bigskip
\subsection*{Mathematical concepts involved in Problem B...}
\begin{itemize}
    \item \textbf{Combining and rearranging polynomial equations:} Moving all terms to one side to combine like terms and reduce the equation to a simpler form.
    \item \textbf{Factorization and simplification:} Factoring out common factors to simplify the polynomial.
    \item \textbf{Substitution method for quartic equations:} Using $y = x^2$ to convert a quartic equation in $x$ into a quadratic equation in $y$.
    \item \textbf{Quadratic formula:} Applying the formula $y=\frac{-b \pm \sqrt{b^2-4ac}}{2a}$ to solve for $y$.
    \item \textbf{Checking validity of roots:} Ensuring that the solutions for $y = x^2$ are nonnegative before taking the square root, which is crucial for identifying valid real solutions.
    \item \textbf{Recognizing extraneous or invalid solutions:} Discarding negative results for $y$ since $y = x^2 \geq 0$.
\end{itemize}

\noindent\hrulefill
\begin{center}
\textbf{************* END OF SOLUTION: B *************}
\end{center}
\hrulefill

\end{document}
